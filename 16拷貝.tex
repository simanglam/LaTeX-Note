\begin{markdown}
#30天 LaTeX 挑戰 Day 16 化學相關

------

LaTeX 也能拿來排版化學相關的事物,但我們需要借用 mhchem 與 chemfig 的力量

```latex
\usepackage{mhchem}
\usepackage{chemfig}
```

##化學式 & 化學反應式

化學式與化學反應式利用 mhchem 提供的`\ce{}`就可以達成了

```latex
\ce{H2O}\\
\ce{H2O2}\\
\ce{NO-}
```

如果需要質量數可以用以下的方式

```latex
\ce{^235_98U}\\
\ce{^2_1H}\\
\ce{^4_2He}
```

`^`代表上標`_`代表下標,也可以打出分子內離子的氧化態

```latex
\ce{Fe^{II}Fe^{III}2O4}
```

計量化學也可以利用相同的方式

```latex
\ce{2H2O}\\
\ce{1/2H2O}\\
\ce{(1/2)H2O}\\
\ce{$n$H2O}
```

化學反應式只需要加入`+`或`->`等等就好了

```latex
\ce{H2O2 -> H2O + O2}
```

如果涉及到沈澱或產生氣體可以利用單獨的`^`跟單獨的小寫 v,可逆反應則更改箭頭的樣式即可

```latex
\ce{^ v}\\
\ce{A <=> B}\\
\ce{CaCO_3 + HCl <=> CaCl_2 v + H_2O + CO_2 ^}
```

如果需要加催化劑,可以用箭頭後加中括號的方式達成

```latex
\ce{A ->[text above][text below] B]}\\
\ce{H2O2 ->[MnO2] H2O + O2 ^}\\
$\ce{x Na(NH4)HPO4 ->[\Delta] (NaPO3)_x + x NH3 ^ + x H2O}$
```

下圖是 mhchem 可以使用的箭頭種類

<圖片>

##結構式

結構式需要借助 chemfig 提供的`\chemfig{}`命令

```latex
\chemfig{H-O-H}
```

你可能會想要調整角度,在`-`後加[]可以解決這個問題,chemfig 可以接受預設角度、絕對角度與相對角度的輸入,預設角度就直接在括號內加入數字,預設是 0 ,之後每增加 1 角度增加 45 度,絕對角度需要在數字前加入`:`,相對角度則是加入`::` 

```latex
\chemfig{A-[1]-[2]-[3]-[4]-[5]-[6]-[7]-[8]}
\chemfig{A-[:45]-[:90]-[:135]-[:180]-[:225]-[:270]-[:315]-[:360]}
\chemfig{A-[::+45]-[::+45]-[::+45]-[::+45]-[::+45]-[::+45]-[::+45]-[::+45]}
```

如果想要畫多邊形可以利用下面的技巧

```latex
\chemfig{C*5(-A-B-C-D-E-F)}\\
\chemfig{[:18]C*5(-A-B-C-D-E-F)}
```

如果你真的想做點什麼複雜的東西,我建議你可以參考奈米小人

\end{markdown}