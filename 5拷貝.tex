\begin{markdown}
#30天 LaTeX 挑戰 Day 5 使用前須知

-----

## 編譯檔案後

在 LaTeX 編譯完後會產生一些中間文件,這些中間文件可以依據功能不同分成以下幾種

* .log 編譯的紀錄檔,所有編譯中出現的問題都可以在這裡找到。
* .aux/.out 用來存放交叉引用的資料。
* .toc/.lof/.lot 目錄、表目錄、圖目錄的生成資料。
* .bbl/.bcf/.blg 與 biblatex 相關的文件。

如果編譯後發現突然多了許多檔案不要驚慌,這些都是 LaTeX 工作所需的檔案。

## 保留字符
下表為 LaTeX 中的保留字符

| 保留字符 | 用途 | 文檔中使用 | 替代指令 |
| :-------- | :---- | :----- | :----- |
|\ |所有命令的開頭|\$\backslash\$|\textbackslash|
|{|開始一個分組|\\{|\textbraceleft|
|}|結束一個分組|\\}|\textbraceright|
|\$|進入數學模式|\\$|\textdollar|
|%|下註解|\\%|NA|
|#|定義巨集|\\#|NA|
|&|表格中的換格標示|\\&|NA|
|\_|數學模式下產生下標字|\\_|\textunderscore|
|^|數學模式下產生下標字|\\^|\textasciicircum|
|~|產生一個空白(禁止斷行)|\\~|\textasciitilde|

大部分的保留字符都可以藉由加一個反斜槓的方式輸出,但唯有反斜槓不行(單個反斜槓是產生空白、兩個反斜槓加在一起是強制換行)只能使用指令 \textbackslash 來輸出。

### 分組

分組是 LaTeX 中的一個概念,可以將其類比為一個 HTML 的 \<p\> 標籤,通常用來限定命令的作用範圍,使用方式也很簡單,就是將想讓命令作用的範圍用{包起來就好了},範例如下。

```latex
\large %更改字型大小
{\large A}A
```
應該會得到下圖的結果
<圖片>
## 命令與環境

命令與環境的差別差在哪裡?相信這是大家最想問的,你可以將命令理解為由反斜槓開始直到數字、保留字符或空白的字串,將環境理解為被 \begin{環境}......\end{環境} 包裹著的區塊,而實際上環境更像是將開啟一個分組與一連串命令加在一起。

```latex
%以下兩種方式在編譯後都會得到一樣的結果
{\large 放大}\begin{large}放大\end{large}
```

### 假空白

LaTeX 的命令有可分為兩種有參數與沒參數的,通常可選參數會被 [] 包圍起來並置於被 {} 包圍起來的的必選參數前。前面提到命令只會在遇到數字、保留字符或空白才會被視為一個整體,這就會導致一個問題,像 `\LaTeX` 這樣沒有必選參數的命令後面必須要接一個空白,但這個空白會被 LaTeX 忽略掉,導致下面的情況

<圖片>

以下有兩個可以解決這個問題的方法

```latex
\LaTeX{}%在命令後接花括號

\LaTeX\ %在命令之後接反斜槓
```

## 處理錯誤

LaTeX 的錯誤有下列三種

* warning
* badbox
* error

第一種是 warning 代表發生了錯誤但並不影響、或不太影響排版結果的問題上,通常這種回去翻 log 檔都會有一些建議,不過不解決也不會什麼大事情發生。

badbox 是 LaTeX 的一個特殊的錯誤類型,這個錯誤類型是來自於 LaTeX 認為排版產出的結果不美觀,而給出的警告,在這類的警告後面通常還會有 badness 來描述到底有多糟糕。

error 則與 warning 相反,其足以使編譯過程停止或導致奇怪的結果,遇到這種問題建議直接向他人詢問,並請附上原始檔與 log 檔的紀錄,以便他人快速釐清問題所在。

\end{markdown}