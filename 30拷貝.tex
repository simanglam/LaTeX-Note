\begin{markdown}
#30天 LaTeX 挑戰 Day 30 繼續前行

-----

30 天的鐵人賽長跑來到了最後一天,今天就不再寫技術相關的內容了,而要來介紹哪裡可以找到更多關於 LaTeX 的資料,

##網頁

以下幾個網頁是我很推薦的資料來源

* CTAN
* Overleaf
* Stack Exchange

CTAN 是 Comprehensive TeX Archive Network 的縮寫,基本上只要是 TeX 有關的資料都會被收藏在此(LaTeX 當然也被包含在內),如果有什麼 package 或使用手冊想要找,甚至是自己寫了一個 package 想要與全世界的 TeX 使用者共享,只要到 CTAN 就對了。

Overleaf 不只提供了線上編譯 LaTeX 的服務,他們也為了推廣 LaTeX 寫了許多的技術文章,最棒的是他們的技術文章是為了初學者而設計的,所以不用怕看不懂,但想當然的內容是用英文寫的。

大部分人應該都聽過 Stack Exchange ,如果你有什麼問題想問,不仿先來這裡看看有沒有人問過。

##書籍

書籍有以下幾本

* The TeX book
* The Not So Short Introduction to LATEX2ε
* 簡單高效 LaTeX
* 大家來學 LaTeX

The TeX Book 是由高德納教授親自編寫的書籍,可以說是

\end{markdown}