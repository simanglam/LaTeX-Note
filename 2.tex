\begin{markdown}
#30天 LaTeX 挑戰 Day 2 編譯引擎、格式、發行版與編輯器(下)
##發行版

TeX 發行版可以說是將編譯引擎、格式與 Pacakge 都集中到一起的集合,通常我們不會單獨下載編譯引擎與格式,而是會直接下載發行版。我所知發行版有以下三種。

* TeX Live
* MiKTeX
* MacTeX

###TeX Live

由 TUG(TeX User Group)維護的發行版,可以說是目前最活躍的 TeX 發行版,但我並不是使用這個發行版,關於使用方法可以參考使用手冊<https://tug.org/texlive/doc.html>。

###MiKTeX

MiKTeX 的哲學是夠用就好(Just Enough TeX),一開始安裝時只需要下載基本的 Package 即可,隨後如果有缺失的 Package 便會在編譯前下載(On The Fly),如果你不想裝龐大的 TeX 發行版可以考慮這個。

###MacTeX

MacTeX 實際上是對 TeX Live 進行改造,加入許多對於 MacOS 系統的優化,適合想在 MacOS 上使用 TeX 卻又不想搗鼓太多的人。

##編輯器

編輯器說穿了就是文本編輯器,如果你對於 LaTeX 非常熟悉,不用下載特別的編輯器也可以進行 LaTeX 的撰寫。不過想當然的,專門為 LaTeX 開發的編譯器一定能讓你事半功倍。


小弟推薦 texmaker,是一個專為編輯 tex 文件所開發的開源軟體,有自動補全命令、顯示文章架構與原始碼和PDF並排的功能,我個人使用下來的經驗是非常美好的。但沒有什麼東西是完美的,texmaker 需要設定比較多的選項才能順利的編譯 tex 文件。

下一章就要教大家如何建構環境了。
\end{markdown}