\begin{markdown}
#30天 LaTeX 挑戰 Day 19 Ti*k*Z

-------

TikZ 是一個強大的繪圖 pacakge,幾乎可以畫出任何東西

##基礎使用

有兩種方式使用 TikZ 繪圖,第一是將命令放在 `\tikz `後,第二是將命令用放在 tikzpicture 環境中

```latex
\tikz 命令
\begin{tikzpicture}
命令
\end{tikzpicture}
```

###直線

在 Tikz 中畫直線需要用到`\draw[選項]`這個命令

```latex
\draw[選項] ...... ;
```

如果想要畫一條直線,只需要在`\draw `的後面加上點的座標標,再由\-\- 連起來即可,最後不要忘記加上分號

```latex
\tikz \draw (0,0)--(2,0)--(2,2);
```

我們可以將起點與終點的座標重疊,達成封閉圖形的效果

```latex
\tikz \draw (0,0)--(2,0)--(2,2)--(0,0);
```

但這樣會有一個問題,如果你加入 [rounded corners] 這個讓原本鋒利的邊角變成圓角的參數時,你就會發現有問題

```latex
\tikz \draw[rounded corners] (0,0)--(2,0)--(2,2)--(0,0);
```

這是因為圖片只有看起來是是封閉的,但 TikZ 並不認為他是一個封閉的圖形,這時候只需要在最後加入 `--cycle` 就可以避免這個問題了

```latex
\tikz \draw[rounded corners] (0,0)--(2,0)--(2,2)--cycle;
```

###矩形

畫矩形也可以用上面的方式畫,但 TikZ 有提供更簡單的方式

```latex
\tikz \draw (0,0) rectangle (2,2);
```

你只需要在 rectangle 後放入對腳座標即可

###圓形、橢圓形、圓弧

畫一個圓也非常簡單,先給出圓心再給出半徑,中間要加入 circle 將兩者隔開即可

```latex
\tikz \draw (0,0) circle (2);
```

在上圖中的是一個圓心在 (0,0) 半徑為 2 的圓形,畫橢圓形與畫圓形類似,只不過需要把 circle 換成 ellipse ,且在指定長軸與短軸時需用 and 將兩者隔開

```latex
\tikz \draw (0,0) ellipse (2 and 1);
```

圓弧則是將 ellipse 換成 arc ,並需要額外指定角度

```latex
\tikz \draw (0,0) arc (0:270:1);
```

上圖畫出了一個由 (0,0) 開始、從 0 度畫到 270 度、半徑為 1 的四分之三圓弧,橢圓形的圓弧當然也可以這樣畫出來

```latex
\tikz \draw (0,0) arc (0:270:1 and 2);
```

###曲線

曲線則是將直線中的`--`換成 .. 就可以畫出貝茲曲線,也因為是畫出貝茲曲線,所以需要加入至少一個控制點

```latex
\tikz \draw (0,0).. controls (1,1) ..(2,0);
```

多個控制點則需要用 and 區隔

```latex
\tikz \draw (0,0).. controls (0.5,1) and (1.5,1) ..(2,0);
```

###格線

格線則是利用 grid 來達成

```latex
\tikz \draw (0,0) grid (0.5,1) and (1.5,1) ..(2,0);
```

###極座標

除了平面座標外也可以指定極座標

```latex
\tikz \draw (0:0)--(45:1);
```

\end{markdown}