\begin{markdown}
#30天 LaTeX 挑戰 Day 10 圖片

------

LaTeX 本身是不能處理圖片的,所以我們需要借用 graphicx 來讓 LaTeX 處理圖片,其實還有另一個可以處理圖片的 package 叫 graphics ,他們兩個像是同一個 package 但用著不同的 interface ,兩個除了可選參數的形式之外,不論是命令還是必選參數都一樣。在這裡介紹的是 graphicx ,如果想要使用 graphics 請參考說明文件<連結>

##基礎

只要使用`\includegraphics{檔案}`就可以將圖片導入文件中了

```latex
\includegraphics{test.png}
```

但這樣會有一個問題,如果今天圖片與 tex 檔不在同一層目錄下就找不到,圖片少的時候還好,但只要圖片一多再加上 LaTeX 編譯時產生的中間文件就足以將你淹沒在茫茫檔案之中,萬幸的是可以利用`\graphicspath{目錄}`來指定圖片檔案的位置。

```latex
\graphicspath{{jpg/}{png/}}
```

這樣 LaTeX 就會自動搜尋 jpg 跟 png 的子目錄了,你可以利用以下的可選參數來調整圖片

|參數|含義|
|-----|-----|
|scale|圖片縮放|
|width|圖片寬度|
|height|圖片高度|
|page|如果是插入多頁pdf,要插入第幾頁|
|draft|啟動草稿模式|

```latex
\includegraphics[scale=0.25]{test.png}\\
\includegraphics[scale=0.5]{test.png}\\
\includegraphics[scale=0.75]{test.png}\\
\includegraphics[draft]{test.png}
```

###除了圖片之外

除了圖片外 graphicx 也提供了以下指令

* `\rotatebox{角度}{文字}`
* `\scalebox{水平縮放}[垂直縮放]{文字}`
* `\reflectbox{文字}`

第一個 `\rotate{}{}` 顧名思義就是旋轉文字

```latex
\rotatebox{0}{文字}\\
\rotatebox{90}{文字}\\
\rotatebox{180}{文字}\\
\rotatebox{270}{文字}\\
```

第二個 `\scalebox{}[]{}` 可以將文字做兩個不同方向的縮放,第三個 `\reflectbox{} `則是讓文字左右翻轉,實際上可以看成 `\scalebox{-1}[1]{文字}`的簡寫

```
\scalebox{1}[1]{文字}\\
\scalebox{2}[1]{文字}\\
\scalebox{1}[2]{文字}\\
\scalebox{2}[2]{文字}\\
\scalebox{-1}[1]{文字}\\
\reflectbox{文字}
```

##浮動體環境

按著以上的方式用了一段時間後,你可能會發現這樣產出的結果並不好看,這時只要將圖片放進 figure 環境即可,LaTeX 就會自動幫挑選好位置插入圖片了

```latex
\begin{figure}
\includegraphics[scale=0.5]{test.png}
\end{figure}
```

你會發現插入圖片的位置跟程式碼的位置不太一樣,這是因為 LaTeX 會自動決定他認為好看的位置,而不是我們想要的位置,這時候可以在 `\begin{figure}[]`後的方括號加入參數

|參數|含義|
|-----|-----|
| h |將圖片放在這裡(不一定跟程式碼一樣,但會相近)|
| t |放在頁面頂部|
| b |放在頁面底部|
| p |為圖片單獨開一頁|
| ! |覆蓋 LaTeX 預設用來決定「好」位置的參數|

```latex
\begin{figure}[h]
\includegraphics[scale=0.5]{test.png}
\end{figure}
```

###文繞圖

如果你想要達成文繞圖的效果,需要借助 wrapfig package 提供的 wrapfig 環境

```latex
%\begin{wrapfigure}{位置}{寬度}
\begin{wrapfigure}{r}{6cm}
\includegraphics[width=5.5cm]{test.png}
\end{wrapfigure}
```

下表是可以使用的位置

|參數|含義|
|-----|-----|
| r |靠右側|
| l |靠左側|
| i |雙面模式下靠書封|
| o |雙面模式下靠書的開口|

\end{markdown}