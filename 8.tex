\begin{markdown}
#30天 LaTeX 挑戰 Day 8 字體與字型

-----

來到了第八天,本篇要講的是 LaTeX 的字體與字型的設定。

##字型

###字體大小

LaTeX 預設內文字體是 10pt 並提供了 11 & 12 pt 可供使用,並且 LaTeX 有預設一些字體大小

|環境|swich|10pt|11pt|12pt|
|-----|-----|-----|-----|-----|
|`\begin{tiny}`|`\tiny` | 5pt | 6pt | 6pt |
|`\begin{scriptsize}`|`\scriptsize` | 7pt | 8pt | 8pt |
|`\begin{footnotesize}`|`\footnotesize` | 8pt | 9pt | 10pt |
|`\begin{small}`|`\small` | 9pt | 10pt | 11pt |
|`預設大小`|`\normalsize` | 10pt | 11pt | 12pt |
|`\begin{large}`|`\large` | 12pt | 12pt | 14pt |
|`\begin{Large}`|`\Large` | 14pt | 14pt | 17pt |
|`\begin{LARGE}`|`\LARGE` | 17pt | 17pt | 20pt |
|`\begin{huge}`|`\huge` | 20pt | 20pt | 25pt |
|`\begin{Huge}`|`\Huge` | 25pt | 25pt | 25pt |

如果想要使用特殊的字體大小可利用`\fontsize{font size}{line skip}\selectfont `


###粗體

使用`\textbf{your word}`或`\bfseries` 來改變字體粗細

```latex
\textbf{Bold} or {\bfseries Bold}
```

###斜體

使用`\textit{your word}`或`\itshape` 來更改文字傾斜。

```latex
\textit{italic} or {\itshape italic}
```

###強調

使用`\emph{Important}`即可

```latex
VERY VERT \emph{IMPORTANT}
```

##字體

由於 LaTeX 支持的字體技術過於久遠,於是這裡所要教學的是在 XeLaTeX 與 LuaLaTeX 的環境下可以用的技巧

###在 xeCJK 上

在 xeCJK 中可以利用`\setCJKmainfont[font features]{font}`來設定主要字體,也可以利用`\newCJKfontfamily[family(可不指定,不指定時等同於switch)]\swich{font}[font features]`來聲明新的字族。

```latex
\setCJKmainfont{TW-Kai}
\newCJKfontfamily\sung{TW-Sung}

標楷體、\sung 宋體
```

###在 luatexja 上

在 luatexja 也是一樣,只不過命令長得不一樣,`\setmainjfont` 與`\newjfontfamily`

```latex
\setmainjfont{TW-Kai}
\newjfontfamily\sung{TW-Sung}

標楷體、\sung 宋體
```
\end{markdown}