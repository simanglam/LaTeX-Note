\begin{markdown}
#30天 LaTeX 挑戰 Day 17 listing

------

如果在 LaTeX 想要輸出程式碼,顯然不可能用 `\textbackslash LaTeX`  如此陽春的方法,所以我們需要特殊的環境或 package 來達成目的。

##verb

最陽春的方法就是利用 LaTeX 內建的 `\verb|......|` 來輸出在行內的程式碼

```latex
用 \verb|\begin{center}| 來將文字置中
```

如果是很重要的程式碼,你想專門為他開一個區塊,可以利用 `\begin{verbatim}......\end{verbatim}` 環境

```latex
\begin{verbatim}
\newcounter{example}
\newenvironment{example}{\refstepcounter{example}\textbf{Example.\theexample}\ }{\\}
\end{verbatim}
```

可以看到這一段程式碼單獨的獨立了出來。

##listings

但這樣並不會幫程式碼上色,如果想要美觀的的輸出程式碼可以借助 listings package 的協助。

```latex
\begin{lstlisting}
\begin{itemize}
\item 1
\item 2
\item 3
\end{itemize}
\end{lstlisting}
```

你會看到上面的例子也沒有改善多少,這是因為我們還沒設定程式碼應該長怎樣。

```latex
\begin{lstlisting}[language={[LaTeX]TeX}, commentstyle=\color{red} ,keywordstyle=\color{blue}, numbers=center]
\begin{itemize}
\item 1
\item 2
\item 3
\end{itemize}
%一個不知道為什麼的列表
\end{lstlisting}
```

這樣看起來就好許多了,可是每一次都打這一大長串也不方便,所以可以利用 `\lstset `來設定默認的參數。

```latex
\lstset{
    language={[LaTeX]TeX},
    basicstyle=\sffamily,
    numbers=left,
    numberstyle=\scriptsize,
    frame=tb,
    tabsize=4,
    commentstyle=\color{blue},
    keywordstyle=\color{red},
    morekeywords={ce,draw,node,foreach,in,chemfig,bond,href,hologo,
    ifthenelse,addplot,addplot3,coordinates}
}
```

language 是設定程式語言的類型,basicstyle 是設定列出來成果的格式,numberstyle 是控制數字的格式,commentstyle 是控制註解的格式,keywordstyle 是控制關鍵字的格式,morekeywords 則是可以自行加入星的關鍵字。

##minted

但上面的方法有一個問題,就是他只能標記出有被設定過的關鍵字,有時候多少會有點不方便,於是有人將 Pygment 與 LaTeX 結合起來做成了 minted 這個 package,在使用 minted 之前請先確保自己的電腦內有 Pygment詳細的下載方式請參考下面這篇文章。<https://clay-atlas.com/blog/2020/02/10/python-chinese-tutorial-package-pygments-code-highlight/>

minted 提供了 `\begin{minted}[參數]{語言}......\end{minted}` 來輸出程式碼

```latex
\begin{minted}{latex}
\begin{itemize}
\item 1
\item 2
\item 3
\end{itemize}
\end{minted}
%一個不知道為什麼的列表
```

可以看到這樣好看很多,如果想要微調輸出格式可以在利用以下的參數。

|參數|含義|
|------|------|
|lineos|顯示程式碼行數|
|bgcolor|背景顏色|
|numbers|顯示程式碼行數(可指定位置)|
|mathescape|可以在 minted 環境中直接輸入數學方程式|
|escapeinside|設定跳拖字符|
|breaklines|可不可將程式碼換行|

```latex
\begin{minted}[lineos, breaklines, mathescape, escapeinside=| |]{latex}
\begin{itemize}
\item 1
\item 2
\item 3
\end{itemize}
$\Sigma_{k=100} \sin(k^\circ)$
|\textcolor{red}{ABC}|
\end{minted}
```

如果想要使用別種配色,Pygment 有內建許多不同的 style 可供選擇,只要使用`\usemintedstyle{style}` 選擇即可。

```latex
%\usemintedstyle{vim}
\begin{minted}[lineos, breaklines, mathescape, escapeinside=| |]{latex}
\begin{itemize}
\item 1
\item 2
\item 3
\end{itemize}
$\Sigma_{k=100} \sin(k^\circ)$
|\textcolor{red}{ABC}|
\end{minted}
```

這樣就可以更改樣式了,詳細的樣式與支持的語言,請參考 Pygment 的官網。

\end{markdown}