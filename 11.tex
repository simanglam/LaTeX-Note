\begin{markdown}
# 30天 LaTeX 挑戰 Day 11 自定義

------

在 LaTeX 中有以下幾種自定義命令、環境的方法

* `\newcommand{cmd}[必選參數]{definition}`
* `\renewcommand{cmd}[必選參數]{definition}`
* `\newenvironment{env}[必選參數]{before env}{after env}`
* `\renewenvironment{env}[必選參數]{before env}{after env}`

##`\newcommand & \renewcommand `

`\newcommand `是拿來自定義命令的,而`\renewcommand ` 則是重新定義現有命令的

```latex
\newcommand{\impotant}[1]{\textcolor{yellow}{#1}}
\important{Important}
```

在上述例子中,第一個花括號是命令,中間的中括號是必選參數的數量,最後一個花括號是命令的定義,這裡是利用了上一篇提到的`\textcolor `將字體顏色變為黃色的,而 #1 則是代表第一個可選參數。

##`\newenvironment & \renewenvironment`

`\newenvironment  & \renewenvironment `與`\newcommand & \renewcommand `的思維一樣,只不過命令要改成環境

```latex
\newenvironment{highlight}{\begin{Large}\color{red}\bfseries}{\end{Large}}
\begin{highlight}
被特別強調的文字
\end{highlight}
```

##編號環境

如果想要讓環境編號就必須利用`\newcounter{名稱}{父計數器} `定義一個新計數器,在使用`\newcounter `定義一個新計數器後, LaTeX 會自動生成`\the名稱 `的命令儲存計數器的值。

```latex
\newcounter{example}
\theexample
```

可以用`\setcounter{計數器}{值}`來設定計數器的值

```latex
\newcounter{example}
\setcounter{example}{40}
\theexample
```

可以用`\stepcounter 或\refstepcounter `將計數器的值加一,兩者的區別在`\refstepcounter `增加的值可以被 label 或 ref 等命令使用。

```latex
\newcounter{example}
第一次試驗\theexample ,\refstepcounter refstepcounter 之後\theexample >
```

範例:

```latex
\newcounter{example}
\newenvironment{example}{\refstepcounter{example}\textbf{\large Example \theexample.}\medskip}{}

```

\end{markdown}