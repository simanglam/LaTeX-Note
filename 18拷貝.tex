\begin{markdown}
#30天 LaTeX 挑戰 Day 18 tcolorbox

------

今天要介紹的是 tcolorbox,它提供了一個簡單的產生高度客製化 color box 的方式。

##基礎使用

tcolorbox 提供了`tcolorbox`這個環境供我們建立 colorbox

```latex
\begin{tcolorbox}
This is a colored box.
\end{tcolorbox}
```

###Style

|參數|含義|
|-----|-----|
|colback|底色|
|colbacklower|下半部分的底色|
|colframe|邊匡顏色|
|coltitle|title 欄的底色|
|colupper|上半部分文字的顏色|
|collower|下半部分文字的顏色|
|coltext|文字顏色|
|subtitle style|title 欄的樣式|
|boxrule|邊匡粗細|
|fonttitle|標題文字的樣式|
|fontupper|上半部分文字的樣式|
|fontlower|下半部分文字的顏色|

需要注意的是`colbacklower`需要搭配其他命令才可使用,之後會介紹到,如果想要設定一個預設值可以利用`\tcbset{}`來完成。

###標題與副標題

可以用`[title=title]`為他加入標題

```latex
\begin{tcolorbox}[title=Title]
This is a colored box with a title.
\end{tcolorbox}
```

也可以用`\tcbsubtitle{Subtitle}`來插入副標題

```latex
\begin{tcolorbox}[title=Title]
This is a colored box with a title.
\tcbsubtitle{Subtitle}
And subtitle.
\end{tcolorbox}
```

###上下分段

如果你想要將一個 box 分成兩段可以利用`\tcblower` 

```latex
\begin{tcolorbox}
Upper Box
\tcblower
Lower Box
\end{tcolorbox}
```

這樣預設會是上下兩段,可以利用`sidebyside`改成左右各佔一半

```latex
\begin{tcolorbox}[sidebyside]
Upper Box
\tcblower
Lower Box
\end{tcolorbox}
```

###更多

但 tcolorbox 可不只有這樣,你可以利用`\tcbuselibrary{}`去調用一些延伸功能,例如調用 skin 可以讓上下兩段的顏色分開設定

```latex
%\tcbuselibrary{skins}
\begin{tcolorbox}[skin=bicolor, sidebyside, colback=gray!30!white,colbacklower=gray!5!white]
Bicolor
\tcblower
Bicolor
\end{tcolorbox}
``` 

當然除了上述的技巧之外,tcolorbox 還有許多用處我沒有講到,如果想要好好的研究可以參考他的使用手冊<連結>。

\end{markdown}