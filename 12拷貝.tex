\begin{markdown}
#30天 LaTeX 挑戰 Day 12 xcolor

------

在前半段的使用中 LaTeX 產出的文字都是黑白的,如果想要讓 LaTeX 變成彩色的,需要加入 xcolor 這個 package

##定義顏色

xcolor 提供了`\definecolor{名字}{模型}{參數}` 命令供定義顏色,xcolor 支持 html, rgb, cmyk 等等的顏色模型,用不同的模型會影響參數的形式,

```latex
\definecolor{cyan1}{rgb}{0, 255, 255}
\definecolor{cyan2}{html}{00FFFF}
\definecolor{cyan3}{cmyk}{255, 0, 0, 0}
```

上面雖然都用不同的顏色模型,但定義出的顏色都是一樣的,xcolor 本身有預定義一些基本顏色,如同下圖所示

<圖片>

除此之外 color 也提供了`svgnames, dvinames, x11names` 這三個選項提供更多預定義好的顏色

```latex
\usepackage[svgnames]{xcolor}
\usepackage[dvinames]{xcolor}
\usepackage[x11names]{xcolor}
```

如果想要讓兩種顏色混合,可以利用`\colorlet{名稱}{混合方式}`來混合兩種顏色


```latex
\colorlet{mycolor1}{yellow!10!red}
\colorlet{mycolor2}{blue!10}
```

mycolor1 會是 10\%的黃色加上90\%紅色,mycolor2 會是10\%藍色加上90\%的白色。

##文字顏色

想要讓文字上色有兩種辦法,一種是利用`\color{}`將更改預設顏色,另一種是利用`\textcolor{顏色}{文字}`小範圍的更改。

```latex
\color{yellow}
Banana\\
\color{red}
Apple \textcolor{blue}{Ocean}
```

如果是想要幫文字上底色,可以使用`\colorbox{顏色}{文字}`上色

```latex
|\colorbox{yellow}{Important}|
\colorbox{yellow}{Important}
```

如果想要邊匡,可以利用`\fcolorbox{邊匡顏色}{底色}{文字}`

```latex
\colorlet{mycolor}{blue!50}
\fcolorbox{red}{yellow}{IMPORTANT}\\
\fcolorbox{blue}{mycolor}{Relax}
```

##背景顏色

背景顏色可以利用`\pagecolor{}`來更改

```latex
\pagecolor{red}
A red paper with some message.
```

\end{markdown}