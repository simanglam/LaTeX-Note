\begin{markdown}
# 30天 LaTeX 挑戰 Day 9 列表與表格

------

##列表

在 LaTeX 中有三種不同的列表環境, 分別是 itemize, enumerate 與 description,這三個在使用上除了輸出結果不同外,其他都是完全相同的。

### itemize

itemize 是最簡單的列表環境

```latex
\begin{itemize}
\item 第一點
\item 第二點
\item 第三點
\end{itemize}
```

只要在環境中利用 `\item` 就可以放置項目符號,如果想要自訂項目符號,只需要像 `\item[]` 這樣指定即可

```latex
\begin{itemize}
\item 第一點
\item[\$]第二點
\item[\#]第三點
\end{itemize}
```

可以看到第二點與第三點的項目符號換成了 \$ 與 \# ,也可以將項目符號換成數字

```latex
\begin{itemize}
\item[1]第一點
\item[2]第二點
\item[2]第三點
\end{itemize}
```

但通常不會有人這樣做,因為可以靠下一個要介紹的列表環境來達成類似的事情。

### enumerate

如同上一段所說, enumerate 的項目符號是連續的數字,如果需要列出有順序的列表,可以考慮使用這個環境。

```latex
\begin{enumerate}
\item 第一點
\item 第二點
\item 第三點
\end{enumerate}
```

如果想要在一個大項目下細分出子項目,可以在 enumerate 環境中再使用一次 enumerate 環境

```latex
\begin{enumerate}
\item 第一點
\begin{enumerate}
\item 第一小點
\item 第二小點
\item 第三小點
\end{enumerate}
\item 第二點
\item 第三點
\end{enumerate}
```

### description

description 環境比較像在說明某些事物時會用到的環境,在使用 `\item `時如果沒有指定項目符號,就會像下圖所示一般

```latex
\begin{description}
\item 什麼都沒有?
\item 什麼都沒有!
\item 什麼都沒有。
\end{description}
```

可以看到原本該有項目符號的地方什麼都沒有,但如果項目符號有被指定,就不會像上面什麼都沒有

```latex
\begin{description}
\item[項目符號] 有東西了?
\item[項目符號] 有東西了!
\item[項目符號] 有東西了。
\end{description}
```

這樣的特性讓他可以用在論文中的符號說明或名詞解釋的地方

```latex
\begin{description}
\item[符號] 解釋
\item[符號] 解釋
\item[符號] 非常非常非常非常長的解釋
\end{description}
```

除此之外,這些列表環境也可以混用,例如下面的例子

```latex
\begin{enumerate}
\item 某化學物質
\begin{itemize}
\item 物理性質
\begin{description}
\item[性質] 解釋
\item[性質] 解釋
\item[性質] 解釋
\end{description}
\item 化學性質
\begin{description}
\item[性質] 解釋
\item[性質] 解釋
\item[性質] 解釋
\end{description}
\end{itemize}
\end{enumerate}
```

可以看到這是一個比較複雜的例子。

##表格

想要在 LaTeX 中使用表格需要利用 tabular 環境

```latex
\begin{tabular}{| c | l r |}
\hline
第一欄 & 第二欄 & 第三欄 \\
\hline
\end{tabular}
```

* 在 `\begin{tabular}` 後的花括號中指定的是欄位及對齊方式,`|` 是代表在這兩欄之間要有分隔線,c, l, r 分別代表置中、置左、置右對齊
* `\hline `是讓 LaTeX 畫一條橫線
* & 是跳到下一欄的的符號
* `\\`是告訴 LaTeX 這一行結束了,要跳到下一行。

如果想指定欄寬可以用 p{寬度} 的方式,但在這種情況下預設是置左對齊

```latex
\begin{tabular}{|p{4cm}|p{2cm}|}
\hline
四公分 & 兩公分 \\
\hline
\end{tabular}
```

但要直接這樣使用會有許多問題,所以我們要將表格放進 table 環境內,原因是在下一篇有提到的浮動體

```latex
\begin{table}
\begin{tabular}{|p{4cm}|p{2cm}|}
\hline
四公分 & 兩公分 \\
\hline
\end{tabular}
\end{table}
```

\end{markdown}