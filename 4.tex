\begin{markdown}
#30天 LaTeX 挑戰 Day 4 中文環境配置

來到了第四天,在將發行版與編譯器都下載好之後終於要進入到使用中文了,以下提供數種支持中文的方式。

##PdfLaTeX + CJK

在 Preamble 中加入`\usepackage{CJKutf8}`並且在需要使用到中文的部分使用`\begin{CJK}{UTF-8}{字體}......\end{CJK}`就可以使用中文了,下面有一個小小的範例

```latex
\usepackage{CJKutf8}
% bsmi = 明體
% bkai = 楷書
\begin{CJK}{UTF-8}{bsmi}
這裡就可以用中文了喔
\end{CJK}
```

但這種方法可以使用的中文字體必須是 TeX 發行版自帶的中文字體,在字體的選擇上有一定的局限性。

##XeLaTeX + fontspec 的土炮用法

在 XeLaTeX 的環境下使用`latex\usepackage{fontspec}`並宣告新的字體`latex\newfont\swich{Font}`,然後就可以在文本區中需要中文的地方用\{\swich 中文\}的方式打出中文了。

##XeLaTeX + CTeX

CTeX 是一套由中國人開發的巨集,但其實他本身並不提供中文支持,只是它會幫你根據你的編譯引擎設定好巨集,除此之外 CTeX 還一並提供了符合中文排版的文件格式、預先定義好的中文字體,但不知道為什麼,這套對 Mac OS 的兼容性並不好。

```latex
\documentclass[•]{•}
\usepackage{ctex}
\begin{document}
我可以用中文了
\end{document}
```

|名稱|用途|
|-----|-----|
|ctexart|簡單的幾頁文件|
|ctexrep|報告|
|ctexbook|書籍|
|ctexbeamer|投影片|

^提供的文件格式

##XeLaTeX + xeCJK

這是 ctex 在 XeLaTeX 的環境下使用的中文支持方案,一些常用的設定如下。

```latex
\usepackage{xeCJK}%匯入巨集
\setCJKmainfont[可選參數]{字體名}%設置主要字體
\setCJKfallbackfont[可選參數]{字體名}%設置備用字體
```

##LuaLaTeX + luatexja

這是必須使用 LuaLaTeX 時才會用到的配置,不然我主要是使用 xeCJK

```latex
%\documentclass[•]{•}
%加-fontspec 才可以設定字體
\usepackage{luatexja-fontspec}
\setmainjfont{Font}
%然後就可以使用中文了
```

##總結

使 LaTeX 支持中文的方法不只一種,可以依照自己的需求尋找最適合的方式,我推薦 XeLaTeX + xeCJK 或 LuaLaTeX + luatexja 的方式,其他就讓它留在歷史的洪流中吧。

\end{markdown}