\begin{markdown}
#30天 LaTeX 挑戰 Day 15 數學(下)

------

今天的內容有涉及到美國數學家協提供的 **amssym, amsfonts** 與 **amsmath** ,若有涉及到這些 package 的應用,我會在下面特別標注,如果沒有標註就是 LaTeX 基本的使用。

##各種的應用

基本的函數都是用反斜槓加函數名稱的方式輸出

||||
|------|------|------|------|
|\sin |\cos |\tan |\cot |
|\arccos |\arcsin |\arctan |\sec |
|\csc |\exp |\log |\deg |
|\lim |\inf |\min |\max |

想要使用其他字體嗎?LaTeX 提供了以下幾種字體

|字體|結果|
|-----|-----|
|`\mathrm{ABCabc123}`|$\mathrm{ABCabc123}$|
|`\mathit{ABCabc123}`|$\mathit{ABCabc123}$|
|`\mathnormal{ABCabc123}`|$\mathnormal{ABCabc123}$|
|`\mathcal{ABCabc123}`|$\mathcal{ABCabc123}$|

數學模式的輸出皆為斜體,可以用 `\mathrm{}` 轉為正體,如果想在數學模式中加粗字體,可以利用 **amsmath** 提供的 `\boldsymbol` 命令

```latex
$
\mu ,\boldsymbol{\mu}\\
\delta ,\boldsymbol{\delta}
$
```

空心粗體則需要 amsfonts 提供的 `\mathbb{}` 命令

```latex
$
x > 1 and x \in \mathbb{R}
$
``` 

如果想要將某個公式的推導過程寫下,可以利用 **amsmath** 提供的 align 環境

```latex
\begin{align}

\end{align}
```

在想要對齊的地方用 & 指定即可,實際上的使用方式就與表格類似,如果不想要編號,使用帶星號的 `align*` 即可

```latex
\begin{align*}

\end{align*}
```

如果需要輸出矩陣,可以使用 `matrix` 環境

```latex
$
\begin{matrix}
3 & 0\\
0 & 3
\end{matrix}
$
```

但這樣就只是一些對齊的的數字,所以我們可以利用以下的方式來輸出含有小括號的矩陣

```latex
\[
\left(\begin{matrix}
2 & 0\\
0 & 2
\end{matrix}\right)
\]
```

或者使用由 **amsmath** 提供的 pmatrix 環境

```latex
\[
\begin{pmatrix}
2 & 0\\
0 & 2
\end{pmatrix}
\]
```

不只是小括號,也可以使用方括號、花括號

```latex
\[
\begin{bmatrix}
2 & 0\\
0 & 2
\end{bmatrix}
\begin{Bmatrix}
2 & 0\\
0 & 2
\end{Bmatrix}
\]
```

甚至是行列式也可以利用這個方法輸出

```latex
\[
\begin{vmatrix}
2 & 0\\
0 & 2
\end{vmatrix}
\begin{Vmatrix}
2 & 0\\
0 & 2
\end{Vmatrix}
\]
```

如果想要輸出聯立方程式,可以利用 **amsmath** 提供的 cases 環境

```latex
\[
\begin{cases}
x &= 1\\
y &= 3x + 9
\end{cases}
\]
```

當然,我所列出的例子只是滄海一粟,實際上還有更多的可能性,但由於我很少利用這部分的功能,所以我只能簡單地把我知道的使用方式全都寫出,更進一步的使用方式可以參考這些文章。

\end{markdown}