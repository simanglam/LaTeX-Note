\begin{markdown}
#30天 LaTeX 挑戰 Day 25 biblatex

------

biblatex 是一個管理參考文獻的 package,他可以幫助我們方便快速的管理參考文獻。

##前置作業

首先我們需要準備 .bib 檔, .bib 檔的基礎形式如下

```latex
@Article{key,
author = {作者},
title = {標題},
journal = {期刊},
year = {年份},
}
```


`@Article` 是宣告參考文獻是期刊中的文章,key 是在文章中引用連結使用的,但通常我們不用親自撰寫 .bib 檔,因為像 Google Scholar 之類的文獻資料庫都會提供 bibtex 的格式。

圖片

上圖是如何在 Google Scholar 取得 .bib 檔的方式。

##基礎使用

在準備好 .bib 檔後就可以開始使用 biblatex 了,首先我們需要告訴 biblatex 我們的 .bib 檔叫什麼名字。

```latex
%\usepackage{biblatex}
\addbibresource{name.bib}
```

利用 `\addbibresource{•}` 告訴 biblatex .bib 檔的名稱後,我們就可以利用 `\cite{key}` 在文章中引用參考文獻了。

```latex
Free software 跟價錢並沒有關係,這裡的 Free 指的是自由。\cite{stallman2002free}
```

如果不是使用 overleaf 的人需要注意,我們需要額外跑一次 bibber 和兩次 LaTeX,順序如下:

1. LaTeX
2. biber
3. LaTeX 
4. LaTeX

這樣就可以引用參考文獻了,但我們還需要用 `\printbibliography` 將有用到的參考資料都列出來。

```latex
\printbibliography
```

這樣所有被引用過的資料就都被列出來了,如果有參考文獻沒有被直接引用,又想要讓他出現在此,需要用 `\nocite{key}` 將他列出來。

```latex
Free software 跟價錢並沒有關係,這裡的 Free 指的是自由。\cite{stallman2002free}
\nocite{key}
\printbibliography
```

如果想將檔案中所有的參考文獻都列出,只需將 key 換成 * 就好了,如果引用了許多文章,但最後在列出時想要分類這一大群的參考文獻時,有兩種方法,第一種是利用 `type=` 來依照參考文獻的類型分類。

```latex
\printbibliography[type=article, title=article]
\printbibliography[type=book, title=book]
```

第二個方法是在撰寫 bib 檔時加入 `keywords` ,以便分類。

```latex
\printbibliography[keyword=LaTeX, title=article]
\printbibliography[keyword=Overleaf, title=book]
```

```latex
@book{stallman2002free,
  title={Free software, free society: Selected essays of Richard M. Stallman},
  author={Stallman, Richard},
  year={2002},
  publisher={Lulu. com},
  keywords={}
}
```

如果想要更進一步的了解 biblatex 到底可以做什麼,可以參考以下幾篇文章。

\end{markdown}