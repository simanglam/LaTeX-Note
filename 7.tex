\begin{markdown}
# 30天 LaTeX 挑戰 Day 7 版面配置

##一些內建的處理

以下是 LaTeX 的文件類別內建的版面配置

|選項|含義|
|-----|-----|
|a4paper|設定紙張大小為a4|
|a5paper|設定紙張大小為a5|
|twoside|雙面模式|
|twocolumn|雙欄模式||
|landscape|將紙張旋轉90度|
|參數|含義|
|paperheight|紙張高度|
|paperwidth|紙張寬度|

選項只需要放在`\documentclass[]{}`的中括號內即可,但下面的參數需要利用`\setlength{參數}{數值}`的方式修改。

```latex
\documentclass[a4paper,landscape]{article}
and
\setlength{\paperheight}{value}
\setlength{\paperwidth}{value}
```

##邊界

邊界可以利用 geometry package 來設定

```latex
\usepackage[key1=value, key2=value]{geometry}
or
\usepackage{geometry}
\geometry{key1=value, key2=value}
```

下表有一些常用的 key

|Key|含義|
|-----|-----|
|top|上邊界|
|bottom|下邊界|
|left|左邊界|
|right|右邊界|
|outter|雙頁模式下的右側邊界|
|inner|雙頁模式下的右側邊界|

##各種距離

這裡要介紹的距離有

* parskip
* parindent
* leftskip
* rightskip
* baselineskip
* lineskip

###parskip

parskip 是指 LaTeX 在兩個段落中加入的空白

```latex
\lipsum[][50]

\lipsum[][50]

\parskip 2cm \lipsum[][50]

\lipsum[][50]
```

可以看到段落間的距離變了

###parindent

parindent 是指段落前的縮進

```latex
\setlength{\parindent}{15pt}
ewjriwerflnioweor
```

但 LaTeX 會將標題後的段落視為引言,引言是不會縮排的

###leftskip & rightskip

這是調整兩邊縮排的

###baselineskip & lineskip

這是跟行距有關的兩個選項,baselineskip 是指兩行字基線的距離,是透過 $font size \times 1.2 \times \linespread{value}$ 得出的,若要在文本區內更改,需要使用 `\selectfont` 命令。

```latex
\setlength{\baselineskip}{12pt}\selectfont
AAAAAA

AAAAAA
\setlength{\baselineskip}{24pt}\selectfont
AAAAAA

AAAAAA
```

lineskip 則是在上下兩條基線超過 baselineskip 時兩行之間的距離,
如果要調整行距,建議使用 setspace package 提供的 `\singlespacing、\onehalfspacing 、\doublespacing` 命令,或者利用 `\linespread{vaule}` 設定行距。

\end{markdown}