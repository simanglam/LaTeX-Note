\begin{markdown}
#30天 LaTeX 挑戰 Day 14 數學(上)

--------

LaTeX 很大的一部分功用是排版科學相關文章,而佔最大宗的還是數學相關的文章,因為 LaTeX 有著平易近人的數學輸入法以及足夠大的談鋞,至今扔是學術界慣用的排版軟體。

##基本概念

最簡單的用法是將方程式用 `$......$` 包起來,這樣可以在行內插入數學方程式

```latex
畢氏定理$C =\sqrt{A^2 + B^2} $
```

但當方程式很複雜、或非常重要,讓你需要為他特別清出空間,好彰顯這個方程式的重要性,這時可以使用 `\[......\]` 把方程式包起來

```latex
畢氏定理:
\[C =\sqrt{A^2 + B^2}\]
相當的重要
```

雖然這兩者在輸入上沒有任何的差別,但在輸出上還是會有些許的不同

```latex
這是隨文數式:$\Sigma^{60}_{k=31}\sin^2k^\circ$
這是展示數式:
\[\Sigma^{60}_{k=31}\sin^2k^\circ\]
```

可以看到上下標的位置有所改變

##基礎使用

先從最簡單的四則運算開始說起,除了乘、除的符號需要用 `\times` 與 `\div` 表示以外,其他的運算子都不需要使用命令來表示。

```latex
$A + B - C \times D \div E = F$
```

如果想要輸出分數,需要使用 `\frac{分子}{分母}` 輸出

```latex
$\frac{a}{b}\\
(\frac{a}{b})^2$
```

上面的例子有一個問題,第二行的括號會看起來太小,這時候可以利用 `\left(......\right)` 來讓 LaTeX 自動調整括號的大小。

```latex
$\left(\frac{a}{b}\right)^2$
```

這樣就完美了
\end{markdown}