\begin{markdown}
#30天 LaTeX 挑戰 Day 27 lualatex

------

Lualatex 是將 Lua 與 TeX 結合在一起,讓改動 TeX 的排版規則時可以不用 TeXing,更詳細的使用需要對 LuaTeX 與 LaTeX 有深刻的認識,目前我的能力還不到這麼深厚,所以我只介紹一些基礎的用法。

## 基礎用法

想要在 LuaLaTeX 裡使用 Lua 需要透過 `\directlua{}` 的協助,這個命令會將花括號中命令轉給 Lua 解釋器,要想讓 Lua 產出的結果可以轉回給 LaTeX 需要用 `tex.sprint`。

```lua
tex.sprint("$\cos(0)$等於".. math.cos(math.rad(0)))
```

執行之後可以看到 $\cos$ 被輸出出來了,這就是基礎的 LuaLaTeX 的用法更進階的也可以將 for 迴圈帶入使用

```lua
\directlua{
tex.sprint("\\begin{tabular}{|c|c|c|}")
tex.sprint("\\hline")
tex.sprint("x & sin(x) & cos(x) \\\\ ")
tex.sprint("\\hline")
for x = 0,180,10 do
	tex.sprint(x .." & ".. math.sin(math.rad(x)) .." & ".. math.cos(math.rad(x)) .." \\\\ ")
	tex.sprint("\\hline")
end
tex.sprint("\\end{tabular}")
}
```

執行後可以看到有表格被產出了,如果有什麼重複性高的指令,也可以用這種方式來節省時間,這是基礎的 LuaLaTeX 的使用方法。

## 進階使用

進階使用我也不會所以在這裡放一個我看到的例子:

```lua
function fadelines(head)
        GLYPH = node.id("glyph")
        WHAT = node.id("whatsit")
        COL = node.subtype("pdf_colorstack")
        colorize = node.new(WHAT,COL)
        cvalue = 0
        for line in node.traverse_id(GLYPH,head) do
            colorize.data = cvalue.." "..1 - cvalue.." .5".." rg"
            node.insert_before(head, line, node.copy(colorize))
            cvalue = math.min(cvalue + .0008, 1)
        end
        return head
    end

    luatexbase.add_to_callback("pre_linebreak_filter", fadelines, "fadelines")
```

這樣產生的結果如下圖:

要達成這種效果,需要對 LuaLaTeX 以及 LaTeX 有著即為深厚的認識。

\end{markdown}