\begin{markdown}
#30天 LaTeX 挑戰 Day 6 文檔結構
---

本篇文章是要介紹 LaTeX 的文檔節構,LaTeX 文檔可以分成兩個大部分:導言區與文本區,這兩個部分是拿來幹嘛的呢?答案都在本篇內。

##導言區

導言區指的是檔案內 \begin{document} 前的部分,通常我們會在這裡引入需要的 package、選擇文件的類別、定義一些需要的參數、命令,你可以簡單的理解為定義模板,或理解為 HTML 的 \<HTML\> 標籤。

```latex
\documentclass[]{}
%導言區
\begin{document}
%文本區
\end{document}
```
在導言區下列兩個命令是最為重要的

* \documentclass[]{}
* \usepackage{}

前者是決定文件的類別,後者是使用巨集,下表與可選的文件類別

| 文件類別 | 用途 |
| ------ | ------ |
| article | 短文章 |
| report | 多章節的長報告 |
| book | 書籍 |
| beamer | 簡報 |

本篇若未特別說明皆是基於 article 類別

|巨集|用途|
|----|----|
|xeCJK|XeTeX 為編譯引擎的環境下提供中文支持|
|xcolor|使 LaTeX 支持多彩|
|mhchem|化學反應式|
|chemfig|化學結構式|
|Geometry|文件版面|
|tikz|繪圖|
|tcolorbox|好看的 color box|
|listings|程式碼展示|
|graphicx|圖片|
|biblatex|參考文獻管理|
∆常用巨集列表

##文本區

文本區才是文章的內容的所在,文章上會顯示的內容都會被打在這裡。

###標題與目錄

在 LaTeX 預定義的文件類別中,有以下幾種的標題格式被預定義好,只要使用這些命令,就可以利用 \tableofcontent 建立目錄,也可以利用 \listoffigre 與 \listoftable 來建立圖目錄與表目錄

|名稱|說明|深度|
|---|---|---|
|\part{}|部|-1 (在 article 為 0)|
|\chapter{}|章|0(在 article 中未被定義)|
|\section{}|節|1|
|\subsection{}|小節|2|
|\subsubsection{}|小小節|3|
|\paragraph{}|段|4|
|\subparagraph{}|小段|5|

深度在 LaTeX 文件類型的定義中是用來決定該不該被 \tableofcontent 編入目錄的,以下有一些有關的指令

```latex
\setcounter{tocdepth}{2}%設定深度

\section*{}%只要加一個星號就會不編號也不編入目錄

\addcontentsline{toc()/lof/lot}{層級}{名稱}%將未編入目錄的標題標入目錄
```
\end{markdown}